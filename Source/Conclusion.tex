\section{Заключение}

Представленная схема устройства для получения пчелиного яда имеет ряд недостатков, поэтому  возможно её дальнейшее совершенствование.

Устройство собрано на устаревших комплектующих. В настоящее время рациональнее будет использовать для решения поставленной задачи микроконтроллер. Это позволит очень гибко управлять работой прибора:
устанавливать любые интервалы задержки и времени работы, параметров формирования пачек импульсов, а также частоту самих импульсов. Кроме этого станет возможным изменять вышеуказанные параметры автоматически в процессе работы устройства.

Применение микроконтроллера также даёт возможность владельцу удалённо управлять работой устройства и отслеживать текущие настройки.
