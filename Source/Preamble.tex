% Настройка стилей для заголовков
\ESKDsectAlign{section}{Center}
\ESKDsectStyle{section}{\large \bfseries}
\ESKDsectSkip{section}{0pt}{12pt}
\ESKDsectStyle{subsection}{\normalsize \bfseries}
\ESKDsectSkip{subsection}{12pt}{6pt}

\usepackage{xecyr}

% Костыль для тире
\newcommand*{\longndash}{-\kern-0.22em-~}
\newcommand*{\longmdash}{-\kern-0.20em-\kern-0.20em-\kern-0.20em-}

% Дополнительная работа с математикой
\usepackage{amsmath,amsfonts,amssymb,amsthm,mathtools}
\usepackage{icomma} % "Умная" запятая: $0,2$ --- число, $0, 2$ --- перечисление

% Чтобы использовать символ градуса (°) - \degree
\usepackage{gensymb}
\usepackage{pdfpages}
% Enables loading of OpenType fonts
\usepackage[cm-default]{fontspec}

% Чтобы использовать << >>, если поддерживает шрифт
\defaultfontfeatures{Mapping=tex-text}
\defaultfontfeatures{Ligatures=TeX}
	
\newcommand{\No}{№} % Костыль для совместимости

% Установка глобального шрифта (FakeBold, FakeSlant - костыль для жирного и курсива)
\setmainfont[
	Path=/home/ivan/MEGA/Study/Схемотехника/Курсовая работа/Fonts/,
	Mapping=tex-text,
	Renderer=Basic,
	AutoFakeBold=1,
	AutoFakeSlant=0.3
]{isocpeur}
%\setmainfont[Path=/home/ivan/MEGA/Study/Схемотехника/Курсовая работа/Fonts/,
%BoldFont=*-bold.ttf,
%ItalicFont=*-italic.ttf,
%BoldItalicFont=*-bold-italic.ttf,
%Mapping=tex-text,
%Renderer=Basic
%]{gost-a}

% Установка начертания подписей
\DeclareCaptionLabelSeparator{longndash}{ \longndash}
\captionsetup[figure]{%
labelsep=longndash,justification=centering,singlelinecheck=false,%
aboveskip=0mm,belowskip=3mm}
\usepackage[font=it]{caption}

% Убирает нумерацию глав в содержании
\renewcommand\thesection{{}} 		

% Нумерация списков цифрами с точкой
\renewcommand{\theenumi}{\arabic{enumi}}
\renewcommand{\labelenumi}{\theenumi.}

% Костыль для добавления в основную надпись
\newcommand{\makedocname}[3]{
	\ESKDputOnStyle{formII}{}{
		\ifnum1=#3\relax
		  \put(86, 6){\parbox[b][23mm][c]{66mm}{\renewcommand{\baselinestretch}{0.8}\centering\normalsize #1}}
		\else
		  \put(86, 12){\parbox[b][23mm][c]{66mm}{\renewcommand{\baselinestretch}{0.8}\centering\normalsize #1}}
		\fi	
		
		\put(86,4){\parbox[b][13mm][c]{66mm}{\renewcommand{\baselinestretch}{0.8}\centering\normalsize #2}}
	}
}