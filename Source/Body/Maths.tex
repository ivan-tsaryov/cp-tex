\section{Расчётная часть}

\subsection*{Расчёт генераторов}

Генераторы G1, G2, G3 выполнены на микросхемах DD1, DD2 и DD5 по стандартным схемам. Все эти микросхемы выполнены по КМОП технологии. Для расчёта частотозадающей цепи таких генераторов воспользуемся следующей формулой $f=\frac{1}{2RC}$. 

\underline{Расчёт генератора на DD1}

Генератор G1 имеет подстройку частоты, что позволяет регулировать время задержки от 1 до 14 часов, поэтому частота генератора должна изменяться от: \[f=\frac{1}{7200..100800~\text{с}}=(0,14..0,01)\cdot10^{-3} \text{Гц}\]

Для получения выходных импульсов задействован выход К176ИЕ12 с коэффициентом деления $2^{15}\cdot60$, поэтому частота задающего генератора должна быть $f= 275,3..19,6 \text{Гц}$. Зададимся величиной емкости конденсатора $C_1 = 47~ \text{нФ}$, тогда: \[R=\frac{1}{2fC_1}=38648..541388~\text{Ом}\]

Данный сопротивление на схеме выполнен в виде последовательно соединенных переменного R2 и постоянного R6 резисторов. В таком случае, при положении движка R2, соответствующего нулевому сопротивлению, сопротивление данной цепи определяет только постоянный резистор. Тогда из номинального ряда выбираем R6 = 39 кОм, а R2 = 500 кОм. 

Мощность данных резисторов определяется по формуле: \[P=\frac{U^2}{R}\]
\[P_{R_2}=\frac{U^2_{\text{пит}}}{R_2} \approx 0,16~\text{мВт}\]
\[P_{R_6}=\frac{U^2_{\text{пит}}}{R_2} \approx 2,08~\text{мВт}\]

\underline{Расчёт генератора на DD2}

Генератор G2 также имеет подстройку частоты для регулировки времени работы от 0,5 до 1 часа, поэтому частота генератора должна изменяться от: \[f=\frac{1}{3600..7200~\text{с}}=(0,28..0,14)\cdot10^{-3} \text{Гц}\]

Для получения выходных импульсов задействован выход К176ИЕ5 с коэффициентом деления $2^{15}$, поэтому частота задающего генератора должна быть $f= 9,17..4,59 \text{Гц}$. Зададимся величиной емкости конденсатора $C_3 = 1~ \text{мкФ}$, тогда: \[R=\frac{1}{2fC_1}=54526..108932~\text{Ом}\]
Тогда из номинального ряда выбираем R9 = 56 кОм, а R10 = 50 кОм. 

Мощность данных резисторов определяется по формуле: \[P=\frac{U^2}{R}\]
\[P_{R_2}=\frac{U^2_{\text{пит}}}{R_2} \approx 1,45~\text{мВт}\]
\[P_{R_6}=\frac{U^2_{\text{пит}}}{R_2} \approx 1,62~\text{мВт}\]

Расчёт генераторов G3 и G4 производится аналогично.

\subsection*{Расчёт преобразователя напряжения}

Как уже было отмечено в разделе "История вопроса", оптимальное напряжение на электродах должно бы в диапазоне от 25 до 35~В. Тогда при напряжении с ШИМ-регулятора в 8,6..12~В, коэффициент трансформации T1 равен k = 2,92. 

Согласно \cite{Krylov_1995}, максимальный ток, протекающий по электродам в ядоприемнике не превышает 100 мкА. С целью уменьшения количества витков на 1~В зададимся большим током вторичной обмотки $I_2=10 \text{мА}$. Тогда мощность вторичной обмотки составляет $P_2=U_2I_2=350~\text{мВт}$.

Далее, принимая КПД трансформатора небольшой мощности, равным около 80 \%, определяем первичную мощность: 
\[P_1=\frac{P_2}{0,8}=437,5~мВт\]

Для определения общей мощности Р трансформатора все мощности, полученные для отдельных обмоток, складываются и общая сумма умножается на коэффициент 1,25, учитывающий потери в трансформаторе: 
\[P=1,25\cdot(P_1+P_2)=984,4~мВт\]

Сила тока, проходящего через первичную обмотку, определяется из общей мощности трансформатора Р:
\[I_1=\frac{P}{U_1}=82~мА\]

Для сердечника из нормальной трансформаторной стали можно рассчитать площадь поперечного сечения S по формуле: 
\[S=1,2\cdot\sqrt{P_1}=0,8~ \text{кв.см}\]

По значению S определяется число витков w' на один вольт. При использовании трансформаторной стали:
\[w'=\frac{50}{S}=62,5~витков\]

Теперь можно рассчитать число витков первичной обмотки, учитывая потери напряжения:
\[w_1=1,1\cdot w'U_1=591 виток\]

Для вторичной обмотки с учетом потерь напряжения число витков равно: 
\[w_2=1,1\cdot w'U_2=1719 витков\]

Диаметры проводов обмотки трансформатора можно определить по формулам:
\[d_1=0,7\cdot\sqrt{I_1}=0,2~ \text{мм}\]
\[d_2=0,7\cdot\sqrt{I_2}=0,07~ \text{мм}\]

Для удобства намотки первичной и вторичной катушек трансформатора провод можно взять марки ПЭВ-1 0,12 мм.

\subsection*{Расчёт временных характеристик устройства}

Среднее время задержки распространения сигнала $t_\text{зд.р.ср.}$ \longndash интервал времени, равный полусумме времени задержки распространения сигнала при включении и выключении ИС:
\[t_\text{зд.р.ср.}=\frac{(I^{1,0}_\text{з.р}+I^{0,1}_\text{з.р})}{2}\]

Значение $t_\text{зд.р.ср.}$ позволяет оценить быстродействие микросхемы и определить допустимую частоту переключений $f_{max}$.

Среднее время задержки распространения сигнала в устройстве можно определить как сумму значений среднего времени задержки распространения сигнала всех логических элементов:
\[t_\text{зд.р.ср.}=\sum\limits_{i=1}^k t_\text{зд.р.ср.i}\]

где $k$ \longndash количество логических элементов.

Среднее время задержки распространения сигнала микросхем устойства:
\begin{itemize}
	\item К176ИЕ5 \longndash 300 нс
	\item К561ЛА7 \longndash 80 нс
	\item К561ЛЕ10 \longndash 135 нс
	\item К561ТМ2 \longndash 285 нс
	\item К176ИЕ12 \longndash 300 нс
	\item К1114ЕУ4 \longndash 150 нс
\end{itemize}

Среднее время задержки распространения сигнала:
\[t_\text{зд.р.ср.}=300+80+135+285+300+150=1250 \text{нс}\]

\subsection*{Расчёт мощности устройства}

Мощность, потребляемая устройством, определяется как суммарная потребляемая мощность всех микросхем устройства:
\[P_\text{пот}=\sum\limits_{i=1}^n P_\text{пот~i}\]

где $n$ \longndash количество ИС в устройстве:
$P_\text{пот~i}$ \longndash потребляемая мощность i-й микросхемы.

Средняя потребляемая мощность \longndash определяется как полусумма  мощностей, потребляемых ИС от источников питания в двух устойчивых состояниях.
\[P_\text{пот.ср.}=U_\text{п}\frac{(I^0_\text{пот}+I^1_\text{пот})}{2}\]

где $U_\text{п}$ напряжение питания,

$I^0_\text{пот}, I^1_\text{пот}$ \longndash токи потребления в состояниях лог. 0 и лог. 1 соответственно.

Средняя потребляемая мощность микросхем устойства:
\begin{itemize}
	\item К176ИЕ5 \longndash 0,0027 Вт
	\item К561ЛА7 \longndash 0,0117 Вт
	\item К561ЛЕ10 \longndash 0,0038 Вт
	\item К561ТМ2 \longndash 0,007 Вт
	\item К176ИЕ12 \longndash 0,0027 Вт
	\item К1114ЕУ4 \longndash 0,18 Вт
\end{itemize}

Средняя потребляемая мощность ИС:
\[P_\text{пот.ср}=0,0027+0,0117+0,0038+0,007+0,0027+0,18=208 \text{мВт}\]

\subsection*{Расчёт стабилизатора напряжения}

Стабилизатор напряжения служит для питания всех цифровых микросхем устройства и должен обеспечивать максимальный ток, который не превышает 1 мА. Для данной цели оптимально применить простейший параметрический стабилизатор, представляющий собой по своей сути специальный делитель напряжения.

Входное напряжение стабилизатора равно 12~В. Выходное \longndash 9~В. Исходя из необходимого напряжения стабилизации, по справочнику \cite{Grigoryev} подбираем стабилитрон Д814Б. Его средний ток стабилизации равен $I_\text{ст} = 5~ \text{мА}$. 

Падение напряжения на гасящем резисторе R3:
\[U_{R1}=U_\text{вх}-U_\text{вых}=12-9=3~ \text{В}\]

Зная ток, протекающий через этот резистор, найдем сопротивление:
\[R_1=\frac{U_{R1}}{I_\text{ст}}=600~ \text{Ом}\]

Выбираем ближайший номинал 620 Ом и определяем минимальную мощность резистора:
\[P_{R1}=U_{R1}I_\text{ст}=0,015~ \text{Вт}\]

Учитывая, что через резистор кроме тока стабилитрона протекает ещё и выходной ток, поэтому выбирают резистор, мощностью не менее, чем в два раза больше вычисленной.