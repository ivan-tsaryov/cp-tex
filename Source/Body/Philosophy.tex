\section{Принцип работы}

Выше было отмечено, что устройство должно работать на пасеке при отсутствии электросети, т. е. должно быть автономным. Принимая также во внимание рассмотренные биологические особенности пчёл, устройство представляет собой электронную систему с ядоприёмником, структурная схема которой представлена на стр.~\pageref{Structural}.

Первичным источником питания в устройстве является автомобильный аккумулятор, к которому подключены ШИМ-регулятор и стабилизатор напряжения. Стабилизатор позволяет получить напряжение величиной +9~В для питания микросхем устройства, а ШИМ-регулятор формирует напряжение, изменяемое в диапазоне от +5 до +12 В, для обеспечения возможности подстройки напряжения на электродах ядоприёмника.

С помощью ключа SA1 устройство можно запустить с задержкой, а с помощью кнопки SB1 \longndash без задержки. Отличие режимов заключается в том, что в первом случае начинает работать генератор импульса запуска, формирующий импульсы с периодом следования 28 часов. В таком случае задержка составляет половину этого периода. Импульс запуска переводит выход триггера в состояние высокого логического уровня, который запускает генератор импульса остановки с периодом следования 1 час.

В то же время триггер запускает мультивибратор и генератор прямоугольных импульсов с регулируемой частотой и скважностью, который модулирует работу мультивибратора, формируя тем самым пачки импульсов частотой 1 кГц. Мультивибратор управляет двумя транзисторными ключами, которые попеременно коммутируют постоянное напряжение, поступающее с ШИМ-регулятора на первичную обмотку трансформатора. Это позволяет получить переменное напряжение прямоугольной формы.

Для повышения полученного переменного напряжения до нужной величины служит повышающий трансформатор, напряжение со вторичной обмотки которого подаётся на электроды ядоприёмника.



