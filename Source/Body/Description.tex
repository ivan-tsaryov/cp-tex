\section{Описание работы устройства}

Работа схемы начинается в момент подключения источника питания к разъёму XP1. 

Стабилизированное напряжение 9~В через резистор R1 поступает на входы сброса генератора G1 и  счётчика D1, держа их в сброшенном состоянии. Также в момент включения конденсатор C5 оказывается разряженным, поэтому на входе R триггера D4 оказывается логическая единица. Конденсатор начинает заряжаться через резистор R4. По мере заряда потенциал на входе R триггера D4 понижается до уровня логического нуля. Данная цепь R5-C4 позволяет получить на входе R триггера импульс, переводящий прямой выход триггера в состояние логического нуля. Этот сигнал держит в сброшенном состоянии генератор G3, а логическая единица с инверсного выхода \longndash генератор G2 и счётчик D2. Также сигнал с этого выхода поступает на элемент логического И-НЕ D5, на выходе которого формируется логическая единица, которая держит генератор G4 в сброшенном состоянии. Поэтому силовые ключи оказываются закрытыми и на выходе схемы на разъёме XP2 напряжение равно нулю. В данном состоянии схема может находиться сколь угодно долго.

Включение устройства в работу производится замыканием ключа SA1. При этом на входах сброса G1 и D1 появляется логический нуль, разрешающий их работу. По истечению T=T\textsubscript{1}/2 часов на выходе делителя D3 появляется логическая единица. Дифференцирующая цепочка R8-C4 обеспечивает появление импульса логической единицы на входе S триггера. При этом на прямом выходе триггера появляется логическая единица, а на инверсном \longndash логический нуль. Это приводит к тому, что генераторы G2, G3 и счётчик D2 начинают работу. Также логический нуль с инверсного выхода триггера поступает на один из входов D5. В моменты, когда сигнал с генератора G3 имеет низкий логический уровень, происходит изменение сигнала на выходе D5 с низкого на высокий логический уровень. При этом начинат работать генератор G4. Противофазные импульсы на выходах генератора G4 попеременно открывают силовые ключи, которые подключают напряжение с ШИМ-регулятора на первичную обмотку трансформатора T1, вследствие чего на его первичной обмотке возникает переменное напряжение. Далее оно усиливается и поступает на выход XP2.

Формирование переменного напряжения на выходе происходит в течение времени равным T\textsubscript{2}/2 часов. По истечению этого времены, на выходе счётчика D2 возникает логическая единица, которая сбрасывает триггер, изменяя состояние его выходов на противоположные. Это приводит к отключению генератора G2 и G3 и сбросу счётчика D2. Также происходит отключение генератора G4 и, следовательно, напряжение на выходе трансформатора T1 станет равным нулю. Схема возвращается в исходное состояние и ожидает появление следующего переднего фронта импульса с выхода D3.

Следует отметить, что возможно включение схемы без дополнительной задержки путем подачи короткого импульса напрямую на вход S триггера при помощи кнопки SB1.

Для регулировки выходного напряжения в схеме ШИМ-регулятора имеется возможность подстройки его выходного напряжения в небольших пределах при помощь переменного резистора R20.
