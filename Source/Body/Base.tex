\section{Выбор элементной базы}

В основе принципа работы данного устройства лежит генерация импульсов с различной частотой и скважностью. Элементы схемы должны иметь высокое быстродействие, а также должны быть экономичными. Именно в таких случаях используются микросхемы серии К176 и К561, КР1561 и 564.

Микросхемы этих серий изготовляются по технологии комплементарных транзисторов структуры металл-диэлектрик-полупроводник (КМДП). Ранее в качестве диэлектрика использовался окисел кремния, поэтому сокращенным обозначением структуры этих микросхем было КМОП.

Основная особенность микросхем КМОП - ничтожное потребление тока в статическом режиме \longndash 0,1..100 мкА. При работе на максимальной рабочей частоте потребляемая мощность увеличивается и приближается к потребляемой мощности наименее мощных микросхем ТТЛ. Микросхемы серий К176, К561 выпускаются в пластмассовых корпусах с двухрядным расположением 14, 16 или 24 штыревых выводов, а микросхемы серии 564 \longndash в корпусах с тем же количеством выводов, расположенных в одной плоскости, в так называемых планарных корпусах. 

Поэтому для построения схемы решено использовать КМОП микросхемы. Теперь определимся с серией.

Номинальное напряжение питания микросхем серии К176 \longndash 9В ± 5 \%, однако они, как правило, сохраняют работоспособность в диапазоне питающих напряжений от 5 до 12 В. Для микросхем серий К561 гарантируется работоспособность при напряжении питания от 3 до 15 В, для КР1561 \longndash от 3 до 18 В.

Диапазон рабочих температур микросхем:
\begin{itemize}
	\item серии К176 от -10 до +70 C;
	\item серий К561 от -45 до +85 C;
	\item серии 564 от -60 до +125 С;
\end{itemize}

Микросхемы серии К561 (564, 1561, 1564) являются более современными по сравнению с серией 176 и превосходят их по всем параметрам. Кроме того, у них более широкий номенклатурный перечень.

Однако для задачи реализации довольно большого времени задержки включения и выключения устройства оптимальнее применить специализированные часовые микросхемы К176ИЕ12 и К176ИЕ5. Для остальных задач будем использовать микросхемы серии К561.

Основные характеристики микросхем серии К176:
\begin{itemize}
	\item P = 10 мкВт/вентиль;
	\item T\SB{зад} = 200 нс;
	\item U\SB{пит} = 5..12В;
\end{itemize}

Основные характеристики микросхем серии К561:
\begin{itemize}
	\item P = 0.4 мкВт/вентиль;
	\item T\SB{зад} = 50 нс;
	\item U\SB{пит} = 3..15В;
\end{itemize}

Благодаря высокому входному сопротивлению (R\SB{вх} > 100 МОм) микросхемы имеют высокую нагрузочную способность К\SB{раз} > 10..30 (количество входов, которые можно подключить к выходу логического элемента, ограничивается только емкостью монтажа; при К\SB{раз} = 10 паразитная емкость нагрузки составляет С\SB{н} = 20 пФ). 

У микросхем все свободные входы логических элементов должны обязательно подключаться к общему проводу или логической "1" (зависит от логики работы). В качестве логической "1" может использоваться напряжение источника питания микросхем. 
Выходные уровни микросхем при работе на однотипные микросхемы практически не отличаются от напряжения питания и потенциала общего провода. Максимальный выходной ток большинства микросхем серий К176, К561 и 564 не стандартизирован и не превышает единиц миллиампер, что несколько затрудняет непосредственное согласование микросхем этих серий с какими-либо индикаторами и микросхемами ТТЛ-серий. 
Для согласования КМОП микросхем с другими сериями с отличными электрическими параметрами используют преобразователи уровня. 

В качестве ШИМ-регулятора используется микросхема К1114ЕУ4, выпускаемая отечественной промышленностью. Она также выпускается рядом зарубежных фирм под разными наименованиями. Например, фирма Texas Instruments (США) выпускает микросхему TL494, фирма SHARP (Япония) \longndash IR3M02, фирма SAMSUNG (Корея) \longndash КА7500, фирма FUJITSU (Япония) \longndash МВ3759 и т.д.

Данные микросхемы специально созданы для построения импульсных источников питания и обеспечивают расширенные возможности при конструировании схем источников питания. Прибор К1114ЕУ4 включает в себя усилитель ошибки, встроенный регулируемый генератор, компаратор регулировки мертвого времени, триггер управления, прецизионный источник опорного напряжения на 5~В и схему управления выходным каскадом. Усилитель ошибки выдает синфазное напряжение в диапазоне от –0,3..(V\SB{сс}-2) В. Компаратор регулировки мертвого времени имеет постоянное смещение, которое ограничивает минимальную длительность мертвого времени величиной порядка 5\%.

Допускается синхронизация вcтроенного генератора, при помощи подключения вывода R к выходу опорного напряжения и подачи входного пилообразного напряжения на вывод С, что используется при синхронной работе нескольких схем импульсных источников питания.

Независимые выходные формирователи на транзисторах обеспечивают возможность работы выходного каскада по схеме с общим эмиттером либо по схеме эмиттерного повторителя. Выходной каскад микросхемы К1114ЕУ4 работает в однотактном или двухтактном режиме с возможностью выбора режима с помощью специального входа. Встроенная схема контролирует каждый выход и запрещает выдачу сдвоенного импульса в двухтактном режиме.

Микросхема гарантирует нормальную работу в диапазоне температур –10..70С.