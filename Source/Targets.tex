\section{Задание}
Цель курсового проекта состоит в следующем: 
\begin{enumerate}
	\item Систематизировать, закрепить и расширить теоретические знания, полученные по дисциплине "Схемотехника"; 
	\item Привить навыки самостоятельного использования полученных в процессе обучения знаний; 
	\item Приобрести опыт в проектировании и анализе цифровых и цифро-аналоговых узлов и устройств; 
	\item Расширить кругозор в области цифровой вычислительной техники; 
	\item Получить практические навыки в оформлении конструкторской документации. 
\end{enumerate}

В курсовом проекте для рассматриваемого устройства, в данном случае речевого сигнализатора, требуется выполнить следующее: 

\begin{enumerate}
		\item Изложить принцип работы устройства; 
		\item Построить структурную схему устройства; 
		\item Построить полную функциональную схему устройства; 
		\item Построить принципиальную схему устройства; 
		\item Дать временные диаграммы работы устройства; 
		\item Привести спецификацию элементной базы устройства; 
		\item Дать расчет технических характеристик устройства; 
		\item Указать пути совершенствования и модификации устройства.
\end{enumerate}

Повествование, в отличие от описания, представляет собой изображение событий или явлений, совершающихся не одновременно, а следующих друг за другом или обусловливающих друг друга. Самый, по-видимому, краткий в мировой литературе пример текста повествования – знаменитый рассказ Цезаря: «Пришёл, увидел, победил». Он ярко и точно передаёт суть повествования – это рассказ о том, что произошло, случилось.
Повествование раскрывает тесно связанные между собой события, явления, действия как объективно происходившие в прошлом. Именно поэтому главное средство такого рассказа – сменяющие друг друга и называющие действия глаголы прошедшего времени совершенного вида. Предложения повествовательных контекстов не описывают действия, а повествуют о них, то есть передают самое событие, действие.
Повествование теснейшим образом связано с пространством и временем. Обозначение места, действия, название лиц и не лиц, производящих действия, и обозначение самих действий – это языковые средства, с помощью которых ведётся повествование.
Стилистические функции повествования разнообразны, связаны с индивидуальным стилем, жанром, предметом изображения. Повествование может быть более или менее объективированным, нейтральным, или, напротив, субъективным, пронизанным авторскими эмоциями.
	